\subchapter{Extra Lab: Building a cross-compiling toolchain using Yocto-1.6}
  {Objective: Learn how build a well tested cross-compiling toolchain using the eglibc C
  library}

\section{This is an optional exercise for home}

If you only want to download a the result of this lab, go to the next chapter.

On a real fast machine like a Dell T7500 with 2 Xeon X5670 (2 x 6 cores/24 threads @ 2.93 GHz/96GB RAM),
\code{bitbake} will run for an hour to complete the build.

On a Core-i7 Quad-Core laptop, like the Dell E6520 with Core i7 2760m/16GB RAM, you should expect 3-4 hours.

Expect a long, long time on a Core 2 Duo with small amount of RAM.

Yocto shows a stack of current executing tasks, as well as to total number of tasks,
so you quickly get an idea about the build time.

After this lab, you will be able to:

\begin{itemize}
\item Generate a modern toolchain for the Beaglebone.
\end{itemize}

\section{Setup}

Go to the \labdir directory.

\section{Install needed packages}

Install the packages needed for this lab, if you havent done this before:

\begin{verbatim}
make prepare
\end{verbatim}

\section{Getting Yocto}

\begin{verbatim}
git clone git://git.yoctoproject.org/poky poky-daisy
\end{verbatim}

Then checkout the daisy branch (Yocto-1.6).

\begin{verbatim}
cd poky-daisy
git checkout -b daisy origin/daisy
\end{verbatim}

\clearpage
\section{Configuring Yocto}

Once you have Yocto installed, you should configure it for your board.

\begin{verbatim}
cd poky-daisy
. oe-init-build-env	build-beaglebone
\end{verbatim}

This will create the \code{build-beaglebone} directory

Check the \code{build-beaglebone/conf} configuration directory.

An important file is \code{local.conf}

\section{Configuring Yocto in local.conf}

You should edit the \code{local.conf} to optimize for your own machine.

Edit the MACHINE variable to set it to the "beaglebone".

This will ensure that the cross compiler will build an ARMv7 toolchain with \code{NEON} support.

\begin{verbatim}
# There are also the following hardware board target machines included for 
# demonstration purposes:
#
MACHINE ?= "beaglebone"
\end{verbatim}

The default is to build a toolchain without libraries for static linking.

We will use static linking, so we need to add support by:

\begin{verbatim}
# Add libraries for static linking
#
IMAGE_INSTALL_append = " eglibc-staticdev"
SDKIMAGE_FEATURES += "staticdev-pkgs dev-pkgs"
\end{verbatim}

A good place is right after the definition of \code{EXTRA_IMAGE_FEATURES}.

If you use a common download directory, you might want to  change the DL\_DIR variable.

\begin{verbatim}
# The default is a downloads directory under TOPDIR which is the build directory.
#
#DL_DIR ?= "${TOPDIR}/downloads"
\end{verbatim}

If you have access with a DVD/USB memory with the tarballs, then you may
want to copy those to the \code{build-beaglebone/downloads} directory to speed up the build.

Change the package mechanism to ipk


\begin{verbatim}
# Package Management configuration
#
# This variable lists which packaging formats to enable. Multiple package backends 
# can be enabled at once and the first item listed in the variable will be used 
# to generate the root filesystems.
# Options are:
#  - 'package_deb' for debian style deb files
#  - 'package_ipk' for ipk files are used by opkg (a debian style embedded package manager)
#  - 'package_rpm' for rpm style packages
# E.g.: PACKAGE_CLASSES ?= "package_rpm package_deb package_ipk"
# We default to rpm:
PACKAGE_CLASSES ?= "package_ipk"
\end{verbatim}


\clearpage
\section{Building the Cross-Compiler}

Yocto has the ability to generate a Software Development Kit (SDK) for an image.

By generating an SDK, you get a cross-compiler with everything needed to build 
further applications outside the Yocto build system.

The SDK contains all the header files for the applications and libraries included in the image. 
\footnote{Some documentation recommends to do {\bf bitbake meta-toolchain} or {\bf bitbake meta-toolchain-sdk}
but for some reason, the static libraries does not seem to be included when you do it this way}

\begin{verbatim}
time bitbake core-image-minimal
\end{verbatim}

followed by

\begin{verbatim}
time bitbake core-image-minimal -c populate_sdk
\end{verbatim}

\code{core-image-minimal} is just that, so you may want to build
a more complete image/toolchain.

\begin{verbatim}
time bitbake core-image-sato
\end{verbatim}

followed by

\begin{verbatim}
time bitbake core-image-sato -c populate_sdk
\end{verbatim}
 
The end result will be a script file in 
\code{~/felabs/sysdev/poky/build/tmp/deploy/sdk}

It is called something similar to:

\code{poky-eglibc-x86_64-core-image-minimal-armv7a-vfp-neon-toolchain-1.6.sh}

(The version number may differ)

\subchapter{Installing a cross-compiling toolchain built by Yocto}
  {Objective: Install a well tested cross-compiling toolchain using the eglibc C
  library}

\section{Getting the Yocto SDK}

If you have done the Extra Lab: \code{Building a cross-compiling toolchain using Yocto}, 
you should have the file:

\code{poky-eglibc-x86_64-core-image-minimal-armv7a-vfp-neon-toolchain-1.6.sh}

or

\code{poky-eglibc-x86_64-core-image-sato-armv7a-vfp-neon-toolchain-1.6.sh}

or similar (the version number may differ)

If not, You can download the toolchain from \code{ftp://ftp.emagii.com/pub/training/tools}
or copy if from the DVD/USB stick, if you got it from your teacher.

Install the latest version, and preferably \code{core-image-sato} over \code{core-image-minimal}.

Make sure this script is executable (Normally it should be).
chnmod
\section{Installing the Cross-Compiler}

\fbox{\begin{minipage}{\textwidth}
{\bfseries
Caution: While the installation script allows you to install the toochain anywhere
it will not work, unless you install it in "/opt/poky/1.6"
}
\end{minipage}}

You normally do not have write access to \code{/opt} so you should
create the install directory before you install the cross-compiler.

\begin{verbatim}
sudo mkdir -p /opt/poky
sudo chown ${USER} /opt/poky
\end{verbatim}

Run the installation script.

You will need to fix LDFLAGS, otherwise it will cause problems with building U-Boot later.

Edit the \code{environment-setup-cortexa8hf-vfp-neon-poky-linux-gnueabi} file in the installation directory:

Comment away the existing definition of LDFLAGS and replace it with an empty defintion.

\begin{verbatim}
export LDFLAGS=""
\end{verbatim}

From the install directory, copy the \code{environment-setup-cortexa8hf-vfp-neon-poky-linux-gnueabi} to \code{toolchain.sh} in the \labdir directory, and make sure it is executable.

If you are running any of the extra labs for creating a toolchain
they may overwrite this file, so beware that you are not running
the wrong cross-compiler. You could name it something else like
\code{yocto-toolchain.sh}

Now you can use the cross compiler by just sourcing this file.

\begin{verbatim}
source toolchain.sh
\end{verbatim}

\section{Simplifying access to the toolchain script}

For easier to access the toolchain script,
you may want to create the \code{~/bin} directory,
and copy the script here.

Add 

\begin{verbatim}
PATH=~/bin:$PATH
\end{verbatim}

to the end of \code{~/.bashrc}

\section{Using the cross-compiler and sudo}

Sometimes, the documentation may ask you do to:

\begin{verbatim}
sudo ${CROSS_COMPILE}<cmd>
\end{verbatim}

This actually does not work, since when you enter superuser mode, you
lose your current environment and CROSS\_COMPILE becomes undefined.

If you run into problems with this, you need to
\begin{verbatim}
sudo su
source toolchain.sh
${CROSS_COMPILE}<cmd>
exit
\end{verbatim}


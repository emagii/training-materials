\subchapter{Setting up an SD-card}{Objectives: Set up an SD-card for use in later labs}

\section{MMC/SD card setup}

The Beaglebone can boot using files from a FAT filesystem on the MMC/SD
card. However, the MMC/SD card must be carefully partitioned, and the
filesystem carefully created in order to be recognized by the ROM
monitor. Here are special instructions to format an MMC/SD card
for the Sitara-based platforms.

First, clean out your system log buffer by \code{sudo dmesg -c}.

Then list all disk devices by \code{ls /dev/sd*}.

Connect your card reader to your workstation, with the MMC/SD
card inside and again list all the disk devices by \code{ls /dev/sd*}.

If you see a new disk device appearing, that will be the new SD-card.
If you see several new devices like \code{ls /dev/sde /dev/sde1 /dev/sde2},
this is because the SD-card has several partitions.

If your PC has an internal MMC/SD card reader, the device may also been
seen as \code{/dev/mmcblk0}, and the first partition as
\code{mmcblk0p1}. \footnote{This is not always the case with internal
MMC/SD card readers. On some PCs, such devices are behind an internal
USB bus, and thus are visible in the same way external card readers
are}. You will see that the MMC/SD card is seen in the same
way by the Beaglebone Black board.

If you still fail to detect the disk, then type the \code{dmesg} command 
to see which device is used by your workstation. 
In case the device is \code{/dev/sde}, you will see something like:

\begin{verbatim}
sd 3:0:0:0: [sde] 3842048 512-byte hardware sectors: (1.96 GB/1.83 GiB)
\end{verbatim}


In the following instructions, we will assume that your MMC/SD card
is seen as \code{/dev/sde} by your PC workstation.

\fbox{\begin{minipage}{\textwidth}
{\bfseries
Caution: read this carefully before proceeding. You could destroy
existing partitions on your PC!

Do not make the confusion between the device that is used by your
board to represent your MMC/SD disk (probably \code{/dev/sda}), and the device
that your workstation uses when the card reader is inserted (probably
\code{/dev/sde}).

So, don't use the \code{/dev/sda} device to reflash your MMC disk from
your workstation. People have already destroyed their Windows
partition by making this mistake.}
\end{minipage}}

You can also run \code{cat /proc/partitions} to list all block devices
in your system. Again, make sure to distinguish the SD/MMC card from the
hard drive of your development workstation!

\clearpage
Type the \code{mount} command to check your currently mounted
partitions. If MMC/SD partitions are mounted, unmount them:

\begin{verbatim}
$ sudo umount /dev/sde1
$ sudo umount /dev/sde2
...
\end{verbatim}

Now, clear possible MMC/SD card contents remaining from previous training 
sessions:

\begin{verbatim}
$ sudo dd if=/dev/zero of=/dev/sde bs=1M count=256
\end{verbatim}

As we explained earlier, the TI Sitara ROM monitor needs special partition geometry settings
to read partition contents. The MMC/SD card must have 255 heads and 63 sectors.

Let's use the \code{cfdisk} command to create a first partition with these settings:

\code{sudo cfdisk -h 255 -s 63 /dev/sde}

In the \code{cfdisk} interface create three primary partitions, starting from the beginning:

\begin{itemize}
\item BOOT: 512 MB size, a \code{Bootable} type and a \code{0C} type (\code{W95 FAT32 (LBA)})
\item ROOTFS:  2048 MB size and a \code{83} type (\code{Linux})
\item DATA: 512 MB size and
a \code{83} type (\code{Linux})
\end{itemize}

Press \code{Write} when you are done.

If you used \code{fdisk} before, you should find \code{cfdisk} much more convenient!

Format the first partition to FAT32, with the \code{boot} label (name):

\begin{verbatim}
sudo mkfs.vfat -n BOOT -F 32 /dev/sde1
\end{verbatim}

Then format the second partition to \code{EXT4}.

\begin{verbatim}
sudo mkfs -t ext4 -L rootfs /dev/sde2
\end{verbatim}

Format the third and last partition to \code{EXT4}.

\begin{verbatim}
sudo mkfs -t ext4 -L data /dev/sde3
\end{verbatim}

Then, remove and insert your card again.

Your MMC/SD card is ready for use.



\subchapter{Tiny embedded system with BusyBox}{Objective: making a tiny yet full featured embedded system}

After this lab, you will:
\begin{itemize}
\item be able to configure and build a Linux kernel that boots on a
  directory on your workstation, shared through the network by NFS.
\item be able to create and configure a minimalistic root filesystem
  from scratch (ex nihilo, out of nothing, entirely hand made...) for
  the Beaglebone Black board
\item understand how small and simple an embedded Linux system can be.
\item be able to install BusyBox on this filesystem.
\item be able to create a simple startup script based on /sbin/init.
\item be able to set up a simple web interface for the target.
\item have an idea of how much RAM a Linux kernel smaller than 1 MB needs.
\end{itemize}

\section{Lab implementation}

While developing a root filesystem for a device, the developer needs
to make frequent changes to the filesystem contents, like modifying
scripts or adding newly compiled programs.

It isn't practical at all to reflash the root filesystem on the target
every time a change is made. Fortunately, it is possible to set up
networking between the development workstation and the target. Then,
workstation files can be accessed by the target through the network,
using NFS.

Unless you test a boot sequence, you no longer need to reboot the
target to test the impact of script or application updates.

\begin{center}
\includegraphics[width=\textwidth]{labs/sysdev-tinysystem-BBB/host-vs-target.pdf}
\end{center}
\clearpage
\section{Setup}

Go to the \code{tinysystem} subdirectory of \labdir.

\section{Kernel configuration}

We will re-use the kernel sources from our previous lab, in
the \code{kernel} subdirectory of \labdir.

In the kernel configuration built in the previous lab, verify that you
have all options needed for booting the system using a root filesystem
mounted over NFS, and if necessary, enable them and rebuild your
kernel.

Make sure your NFS directory is setup.
\clearpage
\section{Booting the system - Skip if you did this in the kernel lab}

First, boot the board to the U-Boot prompt. Before booting the kernel,
we need to tell it that the root filesystem should be mounted over
NFS, by setting some kernel parameters.

If you did not setup your \code{uEnv.txt} file in the SD-Card FAT partition,
do so now:

Add the following U-Boot commands to your uEnv.txt to do so, {\bf netargs in just 1 line}:

\begin{verbatim}
ipaddr=192.168.0.100
serverip=192.168.0.1
loadaddr=0x80200000
fdtaddr=0x80F80000
HOME=/home/ulf
console=ttyO0,115200n8
nfsopts=nolock
netargs=setenv bootargs console=${console} root=/dev/nfs \
  nfsroot=${serverip}:${HOME}/rootfs,${nfsopts} rw ip=${ipaddr}
tftp_kernel=tftp ${loadaddr} zImage
tftp_dtb=tftp ${fdtaddr} am335x-boneblack.dtb
bootkernel=bootz ${loadaddr} - ${fdtaddr}
uenvcmd=run tftp_kernel tftp_dtb netargs bootkernel
mmcboot=echo MMC boot disabled
nandboot=echo NAND boot disabled
\end{verbatim}

Of course, you need to adapt the IP addresses and HOME to your exact network
setup.

You will later need to make changes to the \code{bootargs} value.

Now, boot your system. The kernel should be able to mount the root
filesystem over NFS:

\begin{verbatim}
[    7.467895] VFS: Mounted root (nfs filesystem) readonly on device 0:12.
\end{verbatim}

If the kernel fails to mount the NFS filesystem, look carefully at the
error messages in the console. If this doesn't give any clue, you can
also have a look at the NFS server logs in \code{/var/log/syslog}.

However, at this stage, the kernel may stop because of the below
issue:

\begin{verbatim}
[    7.476715] devtmpfs: error mounting -2
\end{verbatim}

This happens because the kernel is trying to mount the {\em devtmpfs}
filesystem in \code{/dev/} in the root filesystem. To address this,
create a \code{dev} directory under \code{nfsroot} and reboot.

Now, the kernel should complain for the last time, saying that it can't
find an init application:

\footnotesize
\begin{verbatim}
Kernel panic - not syncing: No init found.  Try passing init= option to kernel.
  See Linux Documentation/init.txt for guidance.
\end{verbatim}
\normalsize
\clearpage

\section{Getting the Busybox source code}

Obviously, our root filesystem being mostly empty, there isn't such an
application yet. In the next paragraph, you will add Busybox to your root
filesystem and finally make it usable.

Goto the \labdir directory and then to the \code{tinysystem} subdirectory.

Download the latest BusyBox 1.22.1 release from \code{http://www.busybox.net/downloads}.

You should have run \code{make prepare} initially in \labdir.

If not,install the required packages:
\footnote{If {\bf make gconfig} fails later due to lack of GTK+, you can also try {\bf make menuconfig}}

\begin{verbatim}
sudo apt-get install libglade2-dev libglib2.0-dev libgtk2.0-dev
\end{verbatim}

\section{Unpacking and Patching the Busybox source code}

You will build Busybox in several configurations, each in a separate subdirectory.
Every time you build a new configuration, you need to unpack in the related subdirectory.

Unpack the busybox sources. Then initialize an empty git inside the new directory.

Apply the patches found in the \code{patches} subdirectory of {tinysystem}, in alphabetical order.

Make sure that you do not have any modified or new files if you do \code{git status}.

\section{Configuring Busybox with static libraries}

Ensure that you have the toolchain set-up, and enter the \code{static} subdirectory.

Unpack and patch the busybox source.

You can get the default Busybox configuration (\code{.config}) by:

\begin{verbatim}
make defconfig
\end{verbatim}

This time, Busybox will be built using a predefined configuration file located in the 
\code{static} sub-directory. You should use \code{busybox-1.22.1.config}.
Copy this file to the Busybox configuration file \code{.config} in the Busybox sources.

If you don't use the BusyBox configuration file that we provide, at least,
make sure you build BusyBox statically! Compiling Busybox
statically in the first place makes it easy to set up the system,
because there are no dependencies on libraries. Later on, we will set
up shared libraries and recompile Busybox.

To configure BusyBox, we won't be able to use \code{make xconfig},
which is currently broken in Ubuntu 12.04, because of Qt library
dependencies.

Run \code{make gconfig} to get the configuration window.

Look around in the configuration, and see what options you have.

You will need to set LDFLAGS before the build.

It can be done in the shell, or it can be set in the configuration.

\begin{verbatim}
export LDFLAGS="--static"
\end{verbatim}

Specify the installation directory for BusyBox
\footnote{You will find this setting in
\code{Install Options -> BusyBox installation prefix}.}.
It should be the path to your \code{nfsroot} directory.

Alternatively, you can set \code{_install} directory as a dynamic link
to \code{${HOME}/rootfs}

Exit the configuration, saving changes.

\section{Building Busybox}

Source the \code{toolchain.sh} script and start the build by typing \code{make}

\section{Installing Busybox}

For this lab, we will create our NFS image in \code{tftpboot/busybox.static}

Create the directory and make sure it can be NFS mounted.

If you just type \code{make install} all files will be installed with You as the owner,
but the proper owner for these files is \code{root}.

If you \code{sudo make install}, the files will be installed with the proper owner,
but you will also clear the environment, which means that \code{ARCH} is empty,
and defaults to the host machine \code{x86_64}. This means that all files compiled for ARM, 
will be recompiled, and you get a busybox image for the host, instead of for the \devboard.

The proper way, is to enter superuser modes, setup the environment, and then install

\begin{verbatim}
sudo su
source toolchain.sh
ln -s /tftpboot/busybox.static _install
make install
\end{verbatim}

Verify that you have the right architecture, by entering the \code{/tftpboot/busybox.static/bin}
directory and run  \code{arm-poky-linux-gnueabi-readelf -a busybox}

Also create a \code{dev} directory in \code{nfsroot} if it does not exist.

Copy your kernel and device-tree file to \code{boot} directory of your filesystem.


'sudo chmod -s busybox


\clearpage
\section{Setting up U-Boot for testing Busybox}

Copy the \code{uSetup.txt} from the subdirectory to the SD-card.

\begin{verbatim}
ipaddr=192.168.0.100
serverip=192.168.0.1
loadaddr=0x80200000
fdtaddr=0x88000000

bootdir=/boot
bootfile=zImage

console=ttyO0,115200n8
nfsopts=nolock
optargs=video=HDMI-A-1

setpath=setenv rootpath /tftpboot/${image}
setup_ip=setenv ip ${ipaddr}:${serverip}:${gatewayip}:${netmask}:${hostname}::off

tftp_kernel=run setpath ; tftp ${loadaddr} ${rootpath}/${bootdir}/${bootfile}
tftp_dtb=run setpath    ; tftp ${fdtaddr}  ${rootpath}/${bootdir}/${fdtfile}

mmcargs=setenv bootargs console=${console} ${optargs} root=${mmcroot} \
  rootfstype=${mmcrootfstype} rootwait
mmcrootfstype=squashfs
nandboot=echo NAND boot disabled

bootzImage=if test "${bootfile}" = "zImage"; then bootz ${loadaddr} - ${fdtaddr}; fi
bootuImage=if test "${bootfile}" = "uImage"; then bootm ${loadaddr} - ${fdtaddr}; fi
bootkernel=run bootzImage ; run bootuImage

netargs=run setpath     ; run setup_ip ; setenv bootargs console=${console} \
  ${optargs} root=/dev/nfs nfsroot=${serverip}:${rootpath},${nfsopts} rw ip=${ip}
netboot=echo Booting from network ...; run findfdt; setenv autoload no; \
  run tftp_kernel ; run tftp_dtb ; run netargs; bootm ${loadaddr} - ${fdtaddr}

# image=core-image-minimal
image=busybox.static
br=setenv         image buildroot          ; setenv bootfile zImage ; run bootcmd
dynamic=setenv    image busybox.dynamic    ; setenv bootfile zImage ; run bootcmd
httpd=setenv      image busybox.httpd      ; setenv bootfile zImage ; run bootcmd
static=setenv     image busybox.static     ; setenv bootfile zImage ; run bootcmd

thirdparty=setenv image thirdparty         ; setenv bootfile zImage ; run bootcmd

minimal=setenv    image core-image-minimal ; setenv bootfile uImage ; run bootcmd
base=setenv       image core-image-base    ; setenv bootfile uImage ; run bootcmd
cato=setenv       image core-image-cato    ; setenv bootfile uImage ; run bootcmd

bootcmd=run findfdt ; run tftp_kernel tftp_dtb netargs bootkernel
\end{verbatim}

\section{Testing Busybox}

Try to boot your new system on the board. You should now reach a 
command line prompt, allowing you to execute the commands of your
choice.

\section{Virtual filesystems}

Run the \code{ps} command. You can see that it complains that the
\code{/proc} directory does not exist. The ps command and other
process-related commands use the \code{proc} virtual filesystem to get
their information from the kernel.

From the Linux command line in the target, create the \code{proc}, \code{sys} and
\code{etc} directories in your root filesystem.

Now mount the \code{proc} virtual filesystem.

\begin{verbatim}
mount -t proc nodev /proc
\end{verbatim}

 Now that \code{/proc} is
available, test again the \code{ps} command.

Note that you can also halt your target in a clean way with the \code{halt}
command, thanks to \code{proc} being mounted.

\section{System configuration and startup}

The first userspace program that gets executed by the kernel is
\code{/sbin/init} and its configuration file is \code{/etc/inittab}.

In the BusyBox sources, read details about \code{/etc/inittab} in the
\code{examples/inittab} file.

Then, create a \code{/etc/inittab} file \footnote{Do not use getty as a shell, use /bin/sh}
and a \code{/etc/init.d/rcS} startup script declared in \code{/etc/inittab}. In this startup
script, mount the \code{/proc} and \code{/sys} filesystems.


Reboot and check the contents. They should not be empty.


Any issue after doing this?

\section{Switching to shared libraries}

Take the \code{hello.c} program supplied in the lab \code{data}
directory. Cross-compile it for ARM, dynamically-linked with the
libraries, and run it on the target.

\begin{verbatim}
${CC}          data/hello.c -o hello
${CC} --static data/hello.c -o Hello
\end{verbatim}

Copy the files to the \code{/usr/bin} directory in the \code{nfsroot}  

Try run \code{Hello} on the \devboard. This should work OK.

Then try to run \code{hello}

You will first encounter a \code{not found} error caused by the absence of
the \code{ld-linux-armhf.so.3} executable, which is the dynamic linker
required to execute any program compiled with shared libraries. Using
the find command (see examples in your command memento sheet), look
for this file in the toolchain install directory.

This turns out to be a link to \code{ld-2.19.so}, so copy \code{ld-2.19.so}
to the \code{lib/} directory of the target.

Setup the link:

\begin{verbatim}
ln -s ld-2.19.so /lib/ld-linux-armhf.so.3
\end{verbatim}

Then, running the executable again and see that the loader executes
and finds out which shared libraries are missing. Similarly, find
these libraries in the toolchain and copy them to \code{lib/} on the
target. Setup any needed links.

\begin{verbatim}
ln -s libc-2.19.so libc.so.6
\end{verbatim}

If you get a permission denied error, you may have to set the execute bit
of the libraries.

Once the small test program works, we are going to recompile Busybox
without the static compilation option, so that Busybox takes advantages of the
shared libraries that are now present on the target.

Before doing that, measure the size of the \code{busybox} executable.

\section{Configuring Busybox with shared libraries}

Create the \code{/tftpboot/busybox.httpd} directory and export it via NFS.

Restart the \code{nfs-kernel-server}, so NFS still works.

Go to the \code{httpd} subdirectory.

Use the \code{busybox-1.22.1.config} in \code{httpd} as your config.

Build Busybox with shared libraries, and install it on the target filesystem. 

Copy non busybox parts from the previous build. Dont forget the kernel files.

Create the \code{dynamic} textfile in the toplevel.

Time to boot the system. How can you make sure that 
you are booting from the right NFS export.

What options do you have?

Make sure that the system still boots and see
how much smaller the \code{busybox} executable got.


\begin{verbatim}
ls -l /
\end{verbatim}

You should see the \code{dynamic} file.

\section{Implement a web interface for your device}

Replicate \code{data/www/} to the \code{/www} directory in your target root filesystem.

Now, run the BusyBox http server from the target command line:

\begin{verbatim}
/usr/sbin/httpd -h /www/
\end{verbatim}

It will automatically background itself.

If you use a proxy, configure your host browser so that it doesn't go
through the proxy to connect to the target IP address, or simply
disable proxy usage.  Now, test that your web interface works well by
opening \code{http://192.168.0.100} on the host.

See how the dynamic pages are implemented. Very simple, isn't it?

\section{How much RAM does your system need?}

Check the \code{/proc/meminfo} file and see how much RAM is used by your
system.

You can try to boot your system with less memory, and see whether it
still works properly or not. For example, to test whether 20 MB are
enough, boot the kernel with the \code{mem=20M} parameter. Linux will then
use just 20 MB of RAM, and ignore the rest.

Try to use even less RAM, and see what happens.

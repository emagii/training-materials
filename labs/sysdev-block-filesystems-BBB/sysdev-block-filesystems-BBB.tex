\subchapter{Filesystems - Block file systems}{Objective: configure and
  boot an embedded Linux system relying on block storage}

After this lab, you will be able to:
\begin{itemize}
\item Manage partitions on block storage.
\item Produce file system images.
\item Configure the kernel to use these file systems
\item Use the tmpfs file system to store temporary files
\end{itemize}

\section{Goals}

After doing the {\em A tiny embedded system} lab, we are going to copy
the filesystem contents to the MMC flash drive. The filesystem will be
split into several partitions, and your \devboard will be booted with
this MMC card, without using NFS anymore.

\section{Setup}

Throughout this lab, we will continue to use the root filesystem we
have created in the \nfsroot2 directory, which we will progressively adapt to use block filesystems.

\section{Filesystem support in the kernel}

Your kernel should already be compiled with support for SquashFS and ext3/4.
If not, reconfigure the kernel and recompile and copy to the \code{boot}
directory of \nfsroot2. Also install the kernel modules.

Boot your board with this new kernel and on the NFS filesystem you
used in this previous lab.\footnote{If you didn't do or complete the
  tinysystem lab, you can use the \code{data/rootfs} directory instead.}

\section{Add partitions to the MMC card}

If you followed the setup of the SD-Card, it should be ready for use.
Otherwise go back and setup the SD-card.

Using \code{cfdisk}
\footnote{When you have initialized a partition and want to add more, you don't have to specify headers and sectors again. Just run \code{cfdisk /dev/sdx}} (or \code{cfdisk /dev/mmcblk0x}).

\clearpage
On top of the boot partition, we will need:

\begin{itemize}

\item One partition that will be used for the root filesystem. 
  Due to the geometry of the device, the minimum partition size must be larger than 8 MB,
  but  if we add kernel modules, the size will be much larger.
  We will use 2048 MB.
  Keep the \code{Linux} type for the partition.

\item One partition, that will be used for the data filesystem. 
  Here also, keep the \code{Linux} type for the partition.
  This should be 512 MB.

\end{itemize}

At the end, you should have three partitions: one for the boot, one
for the root filesystem and one for the data filesystem.

\section{Data partition on the MMC disk}

Connect the MMC disk to your board and boot using NFS. 
You should see the MMC partitions in \code{/proc/partitions}.

On the board, create the \code{/media} directory, and mount the third partition
labeled \code{data}.

\begin{verbatim}
mount -t ext4 /dev/mmcblk0p3 /media
\end{verbatim}

Move the contents of the \code{www/upload/files} directory (in your target root filesystem)
into this new partition. The goal is to use the third partition of the
MMC card as the storage for the uploaded images.

Mount this third partition on \code{/www/upload/files}.

Test it by starting \code{httpd} and check with a web-browser.

\begin{verbatim}
/usr/sbin/httpd -h /www
\end{verbatim}

Once this works, modify the startup scripts in your root filesystem
to mount the partition automatically at boot time.

Reboot your target system and with the mount command, check that
\code{/www/upload/files} is now a mount point for the third MMC disk
partition. Also make sure that you can still upload new images, and
that these images are listed in the web interface.

\section{Adding a tmpfs partition for log files}

For the moment, the upload script was storing its log file in
\code{/www/upload/files/upload.log}. To avoid seeing this log file in
the directory containing uploaded files, let's store it in
\code{/var/log} instead.

Add the \code{/var/log/} directory to your root filesystem and modify
the startup scripts to mount a \code{tmpfs} filesystem on this
directory. You can test your \code{tmpfs} mount command line on the
system before adding it to the startup script, in order to be sure
that it works properly.

Modify the \code{www/cgi-bin/upload.cfg} configuration file to store
the log file in \code{/var/log/upload.log}. You will lose your log
file each time you reboot your system, but that's OK in our
system. That's what \code{tmpfs} is for: temporary data that you don't need
to keep across system reboots.

Reboot your system and check that it works as expected.

\section{Making a SquashFS image}

We are going to store the root filesystem in a SquashFS filesystem in
the second partition of the MMC disk.

In order to create SquashFS images on your host, you need to install
the \code{squashfs-tools} package. They should normally be installed.
Now create a SquashFS image of your NFS root directory.

Before creating the filesystem, you should create the \code{boot} directory,
in \nfsroot2 and copy the kernel and device tree file to this directory.

\begin{verbatim}
cd /tftpboot
mksquashfs -all-root -noappend busybox.httpd rootfs.sqfs
\end{verbatim}

Finally, using the \code{dd} command, copy the file system image to
the second partition of the MMC disk.

Be {\bf very careful}, that you do not destroy a hard disk.

\begin{verbatim}
sudo dd if=rootfs.sqfs of=/dev/sdb2
\end{verbatim}

\section{Booting on the SquashFS partition}

In the U-boot shell, configure the kernel command line to use the
second partition of the MMC disk (\code{/dev/mmcblk0p2}) as the 
root file system. (YOU can also modify the \code{uSetup.txt} file.

Remeber that there is a predefined environment which you can use.

Also add the \code{rootwait} boot argument, to wait for the MMC disk to be properly
initialized before trying to mount the root filesystem. Since the MMC
cards are detected asynchronously by the kernel, the kernel might try
to mount the root filesystem too early without \code{rootwait}.

Check that your system still works. Congratulations if it does!

\section{Store the kernel image and DTB on the MMC card}

Finally, copy the \code{zImage} kernel image and DTB to the SD-Card.
What is the easiset?
You can put it in the first partition of the SD card (the partition called \code{boot}),
or you can keep it in the root filesystem. Edit the \code{uSetup.txt}
to adjust the \code{bootcmd} of U-Boot so that it loads the kernel and DTB from the 
SD card instead of loading them through the network.

\section{Booting the easy way}

Reset the board, stop the boot, and try \code{run mmcboot}.

If you set up the correct parameters in \code{uSetup.txt}, it
should work like a charm.
 


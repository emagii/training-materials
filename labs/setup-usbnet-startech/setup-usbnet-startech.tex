\subchapter{Setting up a Startech USB - Ethernet adapter}
{Objective: Get a dedicated network port for communication with the \devboard}

After this lab, you have prepared for a second network port allowing you to nfs-mount the \devboard
root filesystem without changing the network settings of the primary network port.

\section{Prerequisites}

You need a {\bf Startech USB31000SW USB Ethernet} adapter to use the port, but the driver
can be installed without the Adapter.

It is assumed, that your system is setup to build kernel modules.

This means that kernel source build directory must be present

\begin{verbatim}
ls /lib/modules/`uname -r`/build
\end{verbatim}

\section{Preparing to build kernel drivers}

If you did not run \code{make prepare} in \labdir, you need to do this now.

\section{Build and install the kernel driver}

Go to the \labdir directory 

The \code{network/startech} directory contains a tarball with the driver.

The tarball has been downloaded from the Startech website, and will be slightly
modified using a patch to allow it to build.

From the \labdir directory, You can compile and install the driver by:

\begin{verbatim}
sudo make -C network startech-usb
\end{verbatim}

\section{Build and install the kernel driver outside the lab}

Extract the tarball, and enter the source directory.

Apply the patches from the \code{../patches} directory

\begin{verbatim}
make
sudo make install
\end{verbatim}





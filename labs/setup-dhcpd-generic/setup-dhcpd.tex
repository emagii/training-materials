\subchapter{Setting up a DHCP server}
{Objective: Use a DHCP server on your USB Network port}

After this lab, you will be able to easily connect with your \devboard.
Note that all labs in this training assumes that fixed IP addresses
are used, and need to be modified.

\section{Install the DHCP daemon}

\begin{verbatim}
sudo apt-get install isc-dhcp-server
\end{verbatim}



\section{Configuring the DHCP server daemon}

We need to setup

\begin{itemize}
\item /etc/default/isc-dhcp-server

\begin{verbatim}
#Defaults for dhcp initscript
#sourced by /etc/init.d/dhcp
#installed at /etc/default/isc-dhcp-server by the maintainer scripts
#
#This is a POSIX shell fragment
#
#On what interfaces should the DHCP server (dhcpd) serve DHCP requests"
#Separate multiple interfaces with spaces, e.g. “eth0 eth1".
INTERFACES="eth2"
\end{verbatim}
\end{itemize}

The only relevant thing is the \code{INTERFACES} variable, which should be set
to the same port as you use for communication with the \devboard

\clearpage

You also need to setup

\begin{itemize}
\item /etc/dhcp/dhcpd.conf

\begin{verbatim}
#
#Sample configuration file for ISC dhcpd for Debian
#
#Attention: If /etc/ltsp/dhcpd.conf exists, that will be used as
#configuration file instead of this file.
#
#
....
option domain-name “example.org”;
option domain-name-servers ns1.example.org, ns2.example.org;
option domain-name “emagii.com”;
default-lease-time 600;
max-lease-time 7200;
log-facility local7;
subnet 192.168.0.0 netmask 255.255.255.0 {
range 192.168.0.10 192.168.0.100;
option routers 192.168.0.1;
option subnet-mask 255.255.255.0;
option broadcast-address 192.168.0.254;
option domain-name-servers 192.168.0.1;
option ntp-servers 192.168.0.1;
option netbios-name-servers 192.168.0.1;
option netbios-node-type 8;
 ......
}
\end{verbatim}
\end{itemize}


The interesting things here are 

\begin{itemize}

\item subnet: Any range which does not collide with your normal network

\item server adress: We used the fix IP address of the USB Network adapter.

\item range: A range of addresses inside the subnet, but not including the server.

\end{itemize}


\section{Activating the DHCP daemon}

\begin{verbatim}
sudo service isc-dhcp-server restart
\end{verbatim}

Check that it is running:

\begin{verbatim}
sudo netstat -uap
\end{verbatim}

Look for \code{dhcpd}

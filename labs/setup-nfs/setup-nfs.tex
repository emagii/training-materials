\subchapter{Setting up an NFS server}{Objective: Prepare the host for NFS booting}

\section{Install the NFS Server}

\code{make prepare} should have installed the NFS server.
If you do not have an NFS server installed, install it by

\begin{verbatim}
sudo apt-get install nfs-kernel-server
\end{verbatim}

\section{Create the root file system directory}

Create a \code{rootfs} directory in your \$HOME directory. This
directory will later be used to store the contents of our new
root filesystem.

\begin{verbatim}
mkdir -p	$(HOME)/rootfs
chmod 777	$(HOME)/rootfs
\end{verbatim}

\section{Configure the NFS server}

The NFS configuration file (\code{/etc/exports}) needs to be modified.
Edit the file as \code{root} and add the following line: (Replace \code{/home/ulf} with your own home directory)

\begin{verbatim}
/home/ulf/rootfs *(rw,sync,no_root_squash,no_subtree_check)
\end{verbatim}

The format for the line is 
\begin{verbatim}
<directory> <ip-adress>(<options>)
\end{verbatim}

The IP address '*' means that the NFS server will allow any computer to connect.
You can replace it with the IP address of the \devboard.

Make sure that the path and the options are on the same line.
Also make sure that there is no space between the IP address and the NFS
options, otherwise default options will be used for this IP address,
causing your root filesystem to be read-only.

Then, restart the NFS server:

\begin{verbatim}
sudo service nfs-kernel-server restart
\end{verbatim}

alternatively:

\begin{verbatim}
sudo exportfs -av
\end{verbatim}

You can test your NFS setup by:

\begin{verbatim}
sudo mount -t nfs localhost:/home/ulf/rootfs
\end{verbatim}

Replace \code{ulf} with your username.

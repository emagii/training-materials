\subchapter{Extra Lab: Building a cross-compiling toolchain using Yocto}
  {Objective: Learn how build a well tested cross-compiling toolchain using the eglibc C
  library}

\section{This is an optional exercise for home}

After this lab, you will be able to:

\begin{itemize}
\item Generate a modern toolchain for the Beaglebone.
\end{itemize}

\section{Setup}

Go to the \labdir directory.

\section{Install needed packages}

Install the packages needed for this lab, if you havent done this before:

\begin{verbatim}
make prepare
\end{verbatim}

\section{Getting Yocto}

\begin{verbatim}
git clone git://git.yoctoproject.org/poky
\end{verbatim}

Then checkout the dora branch (Yocto-1.5.1).

\begin{verbatim}
cd poky
git checkout -b dora origin/dora
\end{verbatim}

\clearpage
\section{Configuring Yocto}

Once you have Yocto installed, you should configure it for your board.

\begin{verbatim}
. oe-init-build-env
\end{verbatim}

This will create the build directory

Check the \code{build/conf} configuration directory.

An important file is \code{local.conf}

\section{Configuring Yocto in local.conf}

You should edit the \code{local.conf} to optimize for your own machine.

Edit the MACHINE variable to set it to the "beagleboard".
\footnote{Yocto 1.5.1 does not support building a filesystem for the \devboard, as is,
but there will be more boards available in Yocto-1.6. If your board is available, you can use
that as the MACHINE.  \devboard support can be added using extra layers.}

This will ensure that the cross compiler will build an ARMv7 toolchain with \code{NEON} support.

\begin{verbatim}
# There are also the following hardware board target machines included for 
# demonstration purposes:
#
MACHINE ?= "beagleboard"
\end{verbatim}

Try to optimize parallellity for your own host machine by editing \code{PARALLEL_MAKE}
and \code{BB_NUMBER_THREADS}.

\code{PARALLEL_MAKE} describes how many \code{make} jobs should be used
to build an application. 

\code{BB_NUMBER_THREADS} describes how many threads should be used to
build applications. 

The default provided in \code{local.conf} is for a Core 2 Duo host machine.

If you have a Quad Core/Hyperthreaded core, then the default is to conservative,
and the build will be unneccessary slow.
The below values are probably OK, but are depending on disk speed,
RAM size and other things.

This means that the maximum number of tasks are \code{PARALLEL_MAKE * BB_NUMBER_THREADS} 

The goal is to use the CPUs to their maximum, but you also want
to avoid the OS continously switching tasks. You may have to test a couple
of configurations, or just use the guidelines below.

There is a CPU Frequency Utility which you can install in the system tray,
which shows the current CPU frequencies.
If the CPUs are not running at less than full frequencies, you can increase
the two variables.

Since the loading varies over time, there will still be periods during
the build where some of the CPUs are idle.  This happens at the end of 
building the host tools, and at the end of the build, where there are only
a few applications left to build.
\clearpage
\begin{verbatim}
# Make use of SMP:
#   PARALLEL_MAKE specifies how many concurrent compiler threads 
#	are spawned per bitbake process
#   BB_NUMBER_THREADS specifies how many concurrent bitbake tasks will be run
PARALLEL_MAKE     = "-j4"
BB_NUMBER_THREADS = "4"
\end{verbatim}

The default is to build a toolchain without libraries for static linking.

We will use static linking, so we need to add support by:

\begin{verbatim}
# Add libraries for static linking
#
IMAGE_INSTALL_append = " eglibc-staticdev"
SDKIMAGE_FEATURES += "staticdev-pkgs dev-pkgs"
\end{verbatim}

A good place is right after the definition of \code{EXTRA_IMAGE_FEATURES}.

If you use a common download directory, you might want to  change the DL\_DIR variable.

\begin{verbatim}
# The default is a downloads directory under TOPDIR which is the build directory.
#
#DL_DIR ?= "${TOPDIR}/downloads"
\end{verbatim}

If you have access with a DVD/USB memory with the tarballs, then you may
want to copy those to the \code{sources/downloads} directory to speed up the build.

\section{Building the Cross-Compiler}

Yocto has the ability to generate a Software Development Kit (SDK) for an image.

By generating an SDK, you get a cross-compiler with everything needed to build 
further applications outside the Yocto build system.

The SDK contains all the header files for the applications and libraries included in the image. 
\footnote{Some documentation recommends to do {\bf bitbake meta-toolchain} or {\bf bitbake meta-toolchain-sdk}
but for some reason, the static libraries does not seem to be included when you do it this way}

\begin{verbatim}
bitbake bitbake core-image-minimal -c populate_sdk
\end{verbatim}

\code{bitbake} will run for 2-3 hours, on a Core-i7 Quad-Core laptop.
Expect up to 5-7 hours on a Core 2 Duo.
Yocto shows a stack of current executing tasks,
so you quickly get an idea about the time.
 
The end result will be a script file in 
\code{~/felabs/sysdev/poky/build/tmp/deploy/sdk}

It is called \code{poky-eglibc-x86_64-core-image-minimal-armv7a-vfp-neon-toolchain-1.5.1.sh}
or similar.

\subchapter{Installing a cross-compiling toolchain built by Yocto}
  {Objective: Install a well tested cross-compiling toolchain using the eglibc C
  library}

\section{Getting the Yocto SDK}

If you have done the Extra Lab: \code{Building a cross-compiling toolchain using Yocto}, 
you should have the file

\code{poky-eglibc-x86_64-core-image-minimal-armv7a-vfp-neon-toolchain-1.5.1.sh}

If not you will get it from your teacher.

Make sure this script is executable (Normally it should be).

\section{Installing the Cross-Compiler}

Typically the compiler will be installed in \code{/opt/poky/1.5.1}

You normally do not have write access to \code{/opt} so you should
create the install directory before you install the cross-compiler.

You will be asked where to install the cross-compiler, so an alternative 
is to change the installation directory to somewhere where you do have write access.

\begin{verbatim}
sudo mkdir -p /opt/poky
chown <you> /opt/poky
\end{verbatim}

Run the installation script.

Edit the \code{environment-setup-armv7a-vfp-neon-poky-linux-gnueabi} file, 
in the installation directory adding:

\begin{verbatim}
export CROSS_COMPILE=arm-poky-linux-gnueabi-
\end{verbatim}

You also need to fix LDFLAGS, which will cause problems with building U-Boot later.

Comment away the existing definition of LDFLAGS and replace it with an empty defintion.

\begin{verbatim}
export LDFLAGS=""
\end{verbatim}

From the install directory, copy the 
\code{environment-setup-armv7a-vfp-neon-poky-linux-gnueabi} to
\code{toolchain.sh} in the \labdir directory, and make sure it is executable.

If you are running any of the extra labs for creating a toolchain
they may overwrite this file, so beware that you are not running
the wrong cross-compiler. You could name it something else like
\code{yocto-toolchain.sh}

Now you can use the cross compiler by just sourcing this file.

\begin{verbatim}
source toolchain.sh
\end{verbatim}


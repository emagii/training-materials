\subchapter{Training setup}{Download files and directories used in practical labs}

\section{Install lab data}

For the different labs in the training, your instructor has prepared a
set of data (kernel images, kernel configurations, root filesystems
and more). Clone the lab directory to your home directory.

{\small
{\tt
cd \\
git clone https://github.com/emagii/Training-Labs.git felabs
}
}

Lab data are now available in an \labdir directory. 
For each lab there is a directory containing various
data. This directory will also be used as working space for each lab,
so that the files that you produce during each lab are kept separate.

\section{Install extra packages}

You will need to have a number of packages installed on your machine.

Go to the \labdir directory and install required packages using the Makefile.

\begin{verbatim}
make prepare
\end{verbatim}

Since the install requires \code{root} privilegues, you will have to
supply the \code{root} password.

\section{Configure Your lab network}

Edit the \code{host.mk} and change the {\bf SERVER\_IP} and {\bf IPADDR} variables if
they conflict with your standard network.

You are now ready to start the real practical labs!

\clearpage
\section{More guidelines}

Can be useful throughout any of the labs

\begin{itemize}

\item Read instructions and tips carefully. Lots of people make
  mistakes or waste time because they missed an explanation or a
  guideline.

\item Always read error messages carefully, in particular the first
  one which is issued. Some people stumble on very simple errors just
  because they specified a wrong file path and didn't pay enough
  attention to the corresponding error message.

\item Never stay stuck with a strange problem more than 5
  minutes. Show your problem to your colleagues or to the instructor.

\item You should only use the \code{root} user for operations that require
  super-user privileges, such as: mounting a file system, loading a
  kernel module, changing file ownership, configuring the
  network. Most regular tasks (such as downloading, extracting
  sources, compiling...) can be done as a regular user.

\item If you ran commands from a root shell by mistake, your regular
  user may no longer be able to handle the corresponding generated
  files. In this case, use the \code{chown -R} command to give the new
  files back to your regular user.\\
  Example: \code{chown -R myuser:myuser linux-3.4}

\end{itemize}


\subchapter{Setting up a TFTP Server}{Objectives:
Set up TFTP communication with the development workstation.}

\section{Install the TFTPD server}

Later on, we will transfer files from the development workstation to
the board using the TFTP protocol, which works on top of an Ethernet
connection.

Normally, if you did \code{make prepare} before, you should have
the TFTP server running on your system. Otherwise do:

\begin{verbatim}
sudo apt-get install tftpd tftp xinetd
\end{verbatim}

\section{Configuring the TFTPD server}

The configuration file for \code{tftpd} is \code{/etc/xinetd.d/tftp}

An example \code{tftp} file is in \labdir /network
\begin{verbatim}
service tftp
{
        protocol        = udp
        port            = 69
        socket_type     = dgram
        wait            = yes
        user            = nobody
        server          = /usr/sbin/in.tftpd
        server_args     = /tftpboot
        disable         = no
}
\end{verbatim}

Create the configuration file.

The default TFTP directory is \code{/tftpboot}

if it doesnt exist, create it and make it owned by 'nobody'
and make sure that  \code{tftpboot} is easily accessible


\begin{verbatim}
sudo    mkdir -p        /tftpboot
sudo    chown -R nobody /tftpboot
sudo    chmod -R 777    /tftpboot
\end{verbatim}

Restart tftpd by:

\begin{verbatim}
sudo service xinetd restart
\end{verbatim}

\clearpage
\section{Testing the TFTPD server}

Create a file in \code{tftpboot}:

\begin{verbatim}
cd	/tftpboot
echo	111	> testfile.txt
\end{verbatim}

Start tftp and get the file

\begin{verbatim}
cd
tftp	localhost

tftp> get testfile.txt
Sent 5 bytes in 0.0 seconds

tftp> quit

cat testfile.txt

\end{verbatim}



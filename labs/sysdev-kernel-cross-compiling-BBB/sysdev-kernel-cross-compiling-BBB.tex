\subchapter{Kernel - Cross-compiling}{Objective: Learn how to cross-compile a kernel for a Beaglebone Black board.}

After this lab, you will be able to:
\begin{itemize}
\item Set up a cross-compiling environment
\item Configure the kernel Makefile accordingly
\item Cross compile the kernel for the \devboard board
\item Use U-Boot to download the kernel
\item Check that the kernel you compiled starts the system
\end{itemize}

\section{Setup}

Go to the \labdir and from there to the \code{kernel} subdirectory.

Install the required packages if you did not run \code{make prepare} before:

\code{sudo apt-get install libqt4-dev u-boot-tools}

\code{libqt4-dev} is needed for the \code{xconfig} kernel configuration interface, and \code{u-boot-tools}
is needed to build an \code{uImage} kernel image file for U-Boot. We will use the more modern \code{zImage}
so it might not be needed.

You should remove libqt3 development packages, using Synaptic if already present.

\section{Target system}

We are going to cross-compile and boot a Linux kernel for the Beaglebone Black
board.

\section{Kernel sources}

We will re-use the kernel sources downloaded and patched in the
previous lab.

\clearpage

\section{Linux kernel configuration}

Find the proper Makefile target to configure the kernel for the \devboard board 
(hint: the default configuration is not named after the board, but is related to the CPU
name=AM33XX). The configuration files are in \code{arch/arm/configs}.
Find a suitable configuration, and check with the teacher, if it is OK.

Once found, use this target to configure the kernel with the
ready-made configuration through:

\code{ARCH=arm make <board>_defconfig}

Don't hesitate to visualize the new settings by running
\code{make xconfig} afterwards!

This may fail, if libqt3 is installed and used during make.
Deinstall libqt3 headers and retry.

Another reason is that \code{PKG_CONFIG_PATH} is not set up.

\begin{verbatim}
export PKG_CONFIG_PATH=/usr/lib/x86_64-linux-gnu/pkgconfig
\end{verbatim}

\fbox{\begin{minipage}{\textwidth}
{\bfseries
Caution: The \code{toolchain.sh} file changes some critical things, which breaks
the kernel configuration make targets. You may be able to 'make xconfig' by using a eparate terminal where you did not source 'toolchain.sh'.
}
\end{minipage}}

If you have sourced the file, qt development libraries will not be found.

If you get mystic error messages regarding \code{qt-mt.pc}
you may have to reset by \code{make distclean} and redo:

\code{make <board>_defconfig}

In the kernel configuration:

\begin{itemize}

\item In the \code{System Type} menu, make sure that \code{TI AM33XX} is selected.
  Otherwise, you do not have the correct \code{<board>_defconfig}
  We will boot our kernel with a device tree for our board so
  technically speaking, you can leave this option enabled,
  and still boot using a {\em Device Tree}, 

\item As an experiment, let's change the kernel compression from Gzip
  to XZ. This compression algorithm is far more efficient than Gzip,
  in terms of compression ratio, at the expense of a higher
  decompression time. You find XZ in the \code{General Setup} menu.

\item Make sure that SquashFS, ext3 and ext4 filesystems are supported.

\item Make sure that the AM335x Framebuffer is supported

\item Exit the configuration tool, and save the changes.  The result
  is in the \code{.config} file.

\end{itemize}

\section{Cross-compiling environment setup}

To cross-compile Linux, you need to have a cross-compiling
toolchain. We will use the cross-compiling toolchain that we
previously produced, so we just need to make it available in the PATH:

Linux requires some environment variables to be set:

\begin{itemize}
\item \code{ARCH}
\item \code{CROSS_COMPILE}
\end{itemize}

You can also specify them on the command line at every invocation of
\code{make}, i.e: 

\code{make ARCH=... CROSS_COMPILE=... <target>}

Since we created a script file doing this we should just do:

\begin{verbatim}
source ../toolchain.sh
\end{verbatim}

\clearpage
\section{Cross compiling}

You're now ready to cross-compile your kernel. Simply run:

\begin{verbatim}
make -j <n> LOADADDR=0x80008000 zImage
\end{verbatim}

and wait a while for the kernel to compile. Don't forget to use
\code{make -j<n>} if you have multiple cores on your machine!

The \code{LOADADDR} indicates to U-Boot where the kernel image should
be loaded.

Look at the end of the kernel build output to see which file contains
the kernel image. 

Next you need to build the device tree files.

\begin{verbatim}
make dtbs
\end{verbatim}

You can see the Device Tree \code{.dtb} files which got compiled. 
Find which \code{.dtb} file corresponds to your board.

The kernel support dynamically loaded modules, which needs to be built separately.

\begin{verbatim}
make modules
\end{verbatim}

Install the modules in the \code{/tftpboot/rootfs} directory.
We will not use them right now, but we will need them later.
 
\begin{verbatim}
make INSTALL_MOD_PATH=/tftpboot/rootfs modules_install
\end{verbatim}

Check the contents of the \code{/tftpboot/rootfs} directory. There should be a \code{lib} directory.

Copy the \code{zImage} and DTB files to the /tftpboot directory exposed by the TFTP server.

\section{Setting up serial communication with the board}

Plug the Beaglebone Black board on your computer. Start Picocom on
\code{/dev/ttyS0}, or on \code{/dev/ttyUSB0} if you are using a serial
to USB adapter.

\begin{verbatim}
$ picocom -b 115200 /dev/ttyUSB0
\end{verbatim}


Once connected, reset the board, not forgetting to press the {\bf user button}.

Press the {\bf return} key to stop U-Boot from executing the \code{bootcmd}.

You should now see the U-Boot prompt:

\begin{verbatim}
U-Boot #
\end{verbatim}

\clearpage
\section{Load and boot the kernel using U-Boot}

We will use TFTP to load the kernel image to the \devboard board:

You need give a number of commands to boot the kernel, 
Note that you may already have done this for some in the \code{uSetup.txt} file.

Check variables with \code{print}.

\begin{itemize}

\item On the target, set the networking information (use the real ip addresses)	\\
  \code{setenv serverip 192.168.0.1}	\\
  \code{setenv ipaddr 192.168.0.100}

Make sure that the bootargs U-Boot environment variable is set (it
could remain from a previous training session, and this could disturb
the next lab). We will set it to use a ramdisk, but since we do not
have the ramdisk, the kernel will abort.

\begin{verbatim}
setenv bootargs console=ttyO0,115200n8 root=/dev/ram0 rw ramdisk_size=65536 \
  initrd=0x81000000,64M rootfstype=ext2
\end{verbatim}

Make sure the above line is a single line.

\item On the target, load \code{zImage} from TFTP into RAM at address
  0x80200000:\\
  \code{tftp 0x80200000 zImage}

\item Now, also load the DTB file into RAM at address 0x82000000:\\
  \code{tftp 0x80F80000 am335x-boneblack.dtb}

\item Boot the kernel with its device tree:\\
  \code{bootz 0x80200000 - 0x80F80000}

\end{itemize}

You should see Linux boot and finally hang with the following message:

\begin{verbatim}
Kernel panic - not syncing: VFS: Unable to mount root fs on unknown-block(1,0)
\end{verbatim}

This is expected: we haven't provided a working root filesystem for
our device yet.

Reset the board once more ({\bf User Button}!) and wait until the 
\code{bootcmd} has run its course.

\begin{verbatim}
run bootcmd
\end{verbatim}

Now the boot will proceed without a lot of user input, since everything is
handled by the new environment.

If you NFS is setup properly, the kernel may stop because of the below issue:

\begin{verbatim}
[    7.476715] devtmpfs: error mounting -2
\end{verbatim}

This happens because the kernel is trying to mount the {\em devtmpfs}
filesystem in \code{/dev/} in the root filesystem. To address this,
create a \code{dev} directory under \code{nfsroot} and reboot.

Finally, the boot will terminate with the following error message:

\begin{verbatim}
Kernel panic - not syncing: No init found.  Try passing init= option to kernel. See Linux Documentation/init.txt for guidance. 
\end{verbatim}

The reason is that the root file system is empty, and the kernel does not find the \code{init} file.

We will handle this in forthcoming labs.
\clearpage
\section{Automating the boot}

You can automate all this every time the board is booted or reset. 

U-Boot contains a precompiled environment, which will attempt to load
the kernel from an SD-Card.

Remove the micro-SD card and connect it to your host.

Verify the \code{uEnv.txt}.


\begin{verbatim}
ipaddr=192.168.0.100
serverip=192.168.0.1
loadaddr=0x80200000
fdtaddr=0x80F80000
IMAGE=rootfs
console=ttyO0,115200n8
nfsopts=nolock
netargs=setenv bootargs console=${console} root=/dev/nfs \
  nfsroot=${serverip}:/tftpboot/${IMAGE},${nfsopts} rw ip=${ipaddr}
tftp_kernel=tftp ${loadaddr} zImage
tftp_dtb=tftp ${fdtaddr} am335x-boneblack.dtb
bootkernel=bootz ${loadaddr} - ${fdtaddr}
uenvcmd=run tftp_kernel tftp_dtb netargs bootkernel
mmcboot=echo MMC boot disabled
nandboot=echo NAND boot disabled
\end{verbatim}

{\bf Note that the ’netargs’ line should be a single line.}

Remember to run \code{sync} before removing the card.

Reinsert the card in the Beaglebone Black and type

\begin{verbatim}
reset
\end{verbatim}

Do not press the {\bf return} key, to let the \code{bootcmd} execute.

After loading the \code{uEnv.txt} environment, U-Boot is configured to test for
the existence of the \code{uenvcmd} variable, and this will be run, if it exists.

Since \code{uenvcmd} now is defined, the kernel will start to load, but will again abort due to no \code{init}.

\clearpage
\section{Flashing the kernel and DTB in NAND flash (Skip))}

On a system with NAND Flash, you can program the kernel into the flash
after downloading with TFTP. The Beaglebone Black does not have NAND Flash
so this section is for reference only. Read through when you have time, 
but do not perform the commands.

In order to let the kernel boot on the board autonomously we can,
on a board with NAND flash, flash the kernel image and DTB in the NAND flash. 
See the bootloader lab for details about U-boot's \code{nand} command.

After storing the first stage bootloader, U-boot and its environment
variables, we will keep special areas in NAND flash for the DTB
and Linux kernel images:

\begin{center}
  \includegraphics[width=\textwidth]{labs/sysdev-kernel-cross-compiling-BBB/flash-map.pdf}
\end{center}

So, let's start by erasing the corresponding 128 KiB of NAND flash
for the DTB:

\begin{verbatim}
nand erase 0x2e0000 0x20000
        (NAND offset) (size)
\end{verbatim}

Then, let's erase the 5 MiB of NAND flash for the kernel image:

\begin{verbatim}
nand erase 0x300000 0x500000
\end{verbatim}

Then, copy the DTB and kernel binaries from TFTP into memory, using the
same addresses as before.

Then, flash the DTB and kernel binaries:

\begin{verbatim}
nand write 0x81000000 0x2e0000 0x20000
           (RAM addr) (NAND offset) (size)
nand write 0x80000000 0x300000 0x500000
\end{verbatim}

Power your board off and on, to clear RAM contents. We should now be
able to load the DTB and kernel image from NAND and boot with:

\begin{verbatim}
nand read 0x81000000 0x2e0000 0x20000
nboot 0x80000000 0        0x300000
      (RAM addr) (device) (NAND offset)
bootm 0x80000000 - 0x81000000
\end{verbatim}

NAND boot \code{nboot} copies the kernel to RAM, using the \code{uImage} headers
to know how many bytes to copy. You could have used \code{nand read
0x80000000 0x300000 0x500000}, but you would have copied more bytes than
the actual size of your kernel. \footnote{\code{nboot} can save a lot 
of boot time, as it avoids having to copy a pessimistic number of
bytes from flash to RAM}. Modern Linux uses \code{zImage} which is not
supported by \code{nboot}, so the command is not appropriate for newer kernels.

When TFTP downloads a file, it will store the size of the file in the 'filesize'
environment variable. The value can be stored in another variable, I.E: 'kernel\_size'

\begin{verbatim}
tftp 80000000 zImage; 
setenv kernel_size $filesize
tftp 81000000 am335x-boneblack.dtb; 
setenv dtb_size $filesize
\end{verbatim}

This can be used to retreive the files without redundant reads:

\begin{verbatim}
nand read 0x80000000 0x300000 $kernel_size
nand read 0x81000000 0x2e0000 $dtb_size
bootz 80000000 - 81000000
\end{verbatim}

Note that U-boot is not always configured
with \code{nboot} support.

Write a U-Boot script that automates the DTB + kernel download
and flashing procedure. Finally, adjust \code{bootcmd} so that
the board boots using the kernel in Flash.

Now, power off the board and power it on again to check that it boots
fine from NAND flash. Check that this is really your own version of
the kernel that's running.

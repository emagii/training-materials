\subchapter{Kernel sources}{Objective: Learn how to get the kernel
  sources and patch them.}

After this lab, you will be able to:
\begin{itemize}
\item Get the kernel sources from the official location
\item Apply kernel patches
\end{itemize}

\section{Setup}

Goto the \labdir and from there to the \code{kernel} subdirectory.

\section{Get the sources}

Go to the Linux kernel web site (\url{http://www.kernel.org/}) and
identify the latest stable version.

Just to make sure you know how to do it, check the version of the
Linux kernel running on your machine.

We will use \code{linux-3.13.x}, which this lab was tested with.

To practice the patch command later, download the full 3.13
sources. Unpack the archive, which creates a \code{linux-3.13}
directory. Remember that you can use \code{wget <URL>} on the command
line to download files.

Enter the kernel source directory and create a git repo through

\begin{verbatim}
git init
git add .
git commit -m "Initial Commit" -s
\end{verbatim}
\clearpage
\section{Apply patches}

Download the patch file corresponding to the latest 3.13.7 release:
Put the patch file in the parent directory of the kernel source.
This will be used to update the kernel from 3.13 to 3.13.7.

Patches are applied using the \code{patch} program which normally reads from \code{stdin},
but you can supply a filename using the \code{-i} switch.

Normally you have to supply a \code{-p<num>} switch. The patches contains filenames
which are relative to a top directory. The \code{-p<num>} switch will shave off
part of the directory structure of a patch filename. I.E:
 
If \code{-p1} is supplied, \code{/a/lib/mylib.c}  becomes \code{/lib/mylib.c}

If \code{-p2} is supplied, \code{/a/lib/mylib.c}  becomes \code{mylib.c} 

Normally kernel patches requires \code{-p1} so a way to apply patches would be:

\begin{verbatim}
patch -p1 -i path/to/patch-x.y.z
\end{verbatim}

Another way would be

\begin{verbatim}
cat path/to/patch-x.y.z | patch -p1
\end{verbatim}

The downloaded patch is compressed. There is a way to apply a patch
without uncompressing it first into a temporary file.
Can you figure out how?

Without uncompressing it first (!), apply the patch to the Linux source directory.

Use \code{git} to first add all files, and then commit the patch with a reasonable comment.

Check your repo with \code{git status} and \code{git log}. Does it look reasonable?

You can regenerate a patch from the repo. Try:

\begin{verbatim}
git format-patch -1
\end{verbatim}

This will create a file \code{0001-<patch-name>.patch}

View the patch file with \code{vi} or \code{gvim} or any other editor,
to understand the information carried by such a file. 
How are added or removed files described?

Remove the patch file after inspection.

Rename the \code{linux-3.13} directory to \code{linux-3.13.7}.

\subchapter{Setting up the TFTPD-HPA server}{Objectives:
Set up TFTP communication with the development workstation.}

\section{Install the TFTPD-HPA server}

Later on, we will transfer files from the development workstation to
the board using the TFTP protocol, which works on top of an Ethernet
connection.\footnote{the tftpd-hpa server is more modern than the
tdtpd server, but it has problems on Ubuntu 12.04, 
so it is not recommended to use it if your workstation has this installation}

To start with, install and configure a TFTP server on your development
workstation, as detailed in the bootloader slides.

Normally, if you did \code{make prepare} before, you should have
the TFTP server running on your system. Otherwise do:

\begin{verbatim}
sudo apt-get install tftpd-hpa tftp-hpa
\end{verbatim}

\section{Configuring the TFTPD-HPA server}

The configuration file for \code{tftpd} is \code{/etc/init/tftpd-hpa.conf}
which uses \code{/etc/default/tftpd-hpa}.

The default TFTP directory is \code{/var/lib/tftpboot}

You may want to change this to:

\begin{verbatim}
TFTP_DIRECTORY="/tftpboot"
\end{verbatim}

and make sure that  \code{tftpboot"} is easily accessible

\begin{verbatim}
sudo    mkdir -p        /tftpboot
sudo    chown -R nobody /tftpboot
sudo    chmod -R 777    /tftpboot
\end{verbatim}

Restart tftpd by:

\begin{verbatim}
service tftpd-hpa restart
service tftpd-hpa status
\end{verbatim}

\clearpage
\section{Testing the TFTPD-HPA server}

Create a file in \code{tftpboot}:

\begin{verbatim}
cd	/tftpboot
echo	111	> testfile.txt
\end{verbatim}

Start tftp and get the file

\begin{verbatim}
cd
tftp	localhost

tftp> get testfile.txt
Sent 5 bytes in 0.0 seconds

tftp> quit

cat testfile.txt

\end{verbatim}



\subchapter{Extra Lab: Building a cross-compiling toolchain using Yocto}
  {Objective: Learn how build a well tested cross-compiling toolchain using the eglibc C
  library}

\section{This is an optional exercise for home}

After this lab, you will be able to:

\begin{itemize}
\item Generate a modern toolchain for the Beaglebone.
\end{itemize}

\section{Setup}

Go to the \labdir directory.

\section{Install needed packages}

Install the packages needed for this lab:

\begin{verbatim}
make prepare
\end{verbatim}

\section{Getting Angstrom}

Angstrom is a Yocto compliant distribution supporting the Beaglebone

\begin{verbatim}
git clone https://github.com/Angstrom-distribution/setup-scripts.git	Angstrom-1.4
\end{verbatim}

Then checkout the Angstrom-1.4 branch.
The "master" branch is based on Angstrom-1.3 and has some known bugs in the kernel.
The Angstrom-1.4 is currently (Feb 2014) the best branch.
The Anstrom-1.5 branch is still in development.

\begin{verbatim}
cd	Angstrom-1.4
git checkout -b v2013.06 origin/angstrom-v2013.06-yocto1.4
\end{verbatim}

\section{Configuring Angstrom}

Check the \code{sources} directory. It contains the \code{layers.txt} file which is a
list of \code{meta-layers} which will be used to build everything.

Also check the \code{conf} directory. Which files are present.

Once you have Angstrom installed, you should configure it for your board.

\begin{verbatim}
./oebb.sh config beaglebone
\end{verbatim}

This will setup the encironement file \code{environment-angstrom-v2013.06}
and will also add some files to the \code{conf} directory.

Check the \code{sources} directory again. Now all the layers described in \code{layers.txt}
are downloaded.

If you have access with a DVD/USB memory with the tarballs, then you may
want to copy those to the \code{sources/downloads} directory to speed up the build.

You also want now want to checkout the \code{conf} directory again.

Some files have been added, which?

Have a look at the files. 

\begin{itemize}
\item \code{auto.conf} contains the machine description.

\item \code{bblayers.conf} is used to generate the BBLAYERS variable
   which descripes the search path for recipes.

\item \code{local.conf} Main configuration file

\item \code{sanity_info} Used for check duiring the build

\item \code{site.conf} describes the location for \code{build} directories

\end{itemize}

You should edit the \code{local.conf} to optimize for your own machine.

Make sure that you generate a tar.xz file
\begin{verbatim}
./oebb.sh config beaglebone
# What kind of images do we want?
IMAGE_FSTYPES_append = " tar.xz"
\end{verbatim}

Also try to optimize parallellity for your own machine.
If you have a Quad Core/Hyperthreaded core, then the default, 
the below values are probably OK, but depeding on disk speed,
RAM size and other things, the optimal value therefore is depending
on your build machine.

\code{PARALLEL_MAKE} describes how many \code{make} jobs should be used
to build an application. 

\code{BB_NUMBER_THREADS} describes how many threads should be used to
build applications. 

This means that the maximum number of tasks are \code{PARALLEL_MAKE * BB_NUMBER_THREADS} 

The goal is to use the CPUs to their maximum, but you also want
to avoid the OS continously switching tasks. You may have to test a couple
of configurations, or just use the guidelines below.

\begin{verbatim}
# Make use of SMP:
#   PARALLEL_MAKE specifies how many concurrent compiler threads are spawned per bitbake process
#   BB_NUMBER_THREADS specifies how many concurrent bitbake tasks will be run
PARALLEL_MAKE     = "-j6"
BB_NUMBER_THREADS = "12"
\end{verbatim}


\section{Building the Cross-Compiler}

A side functionality of Angstrom/Yocto is the ability to generate
a Software Development Kit (SDK) for an image.

The SDK contains a cross-compiler and all the header files for the applications
and libraries included in the image. By generating an SDK, you have a crosscompiler
with everything needed to build further applications outside the Yocto build system.

\begin{verbatim}
./environment-angstrom-v2013.06
bitbake -c populate_sdk console-image
\end{verbatim}

\code{bitbake} will run for 3-5 hours, on a Core-i7 Quad-Core laptop
Expect up to 5-7 hours on a Core 2 Duo.

The end result will be a script file in 
\code{~/felabs/sysdev/Angstrom-1.4/build/tmp-angstrom_v2013_06-eglibc/deploy/sdk}

It is called \code{angstrom-eglibc-x86_64-armv7ahf-vfp-neon-v2013.06-toolchain.sh}
Make sure it is executable.

\subchapter{Installing a cross-compiling toolchain built by Yocto/Angstrom}
  {Objective: Install a well tested cross-compiling toolchain using the eglibc C
  library}

\section{Getting the Yocto SDK}

If you have done the Extra Lab: \code{Building a cross-compiling toolchain using Yocto}, 
you should have the file

\code{angstrom-eglibc-x86_64-armv7ahf-vfp-neon-v2013.06-toolchain.sh}

If not you will get it from your teacher.

Make sure this script is executable.

\section{Installing the Cross-Compiler}

Typically the compiler will be installed in \code{/usr/local/oecore-x86_64}

You normally do not have write access to \code{/usr/local} so you should
create the install directory before you install the cross-compiler.

Dustin installation, You will be asked where to install the cross-compiler, so an alternative 
is to change the installation directory to somewhere where you do have write access.

\begin{verbatim}
sudo mkdir -p /usr/local/oecore-x86_64
chown <you> /usr/local/oecore-x86_64
\end{verbatim}

Run the installation script.

Edit the \code{environment-setup-armv7ahf-vfp-neon-angstrom-linux-gnueabi} file, 
in the installation directory adding:

\begin{verbatim}
export CROSS_COMPILE="arm-angstrom-linux-gnueabi-"
\end{verbatim}


From the install directory, copy the 
\code{environment-setup-armv7ahf-vfp-neon-angstrom-linux-gnueabi} to
\code{toolchain.sh} in the \labdir directory, and make sure it is executable.

If you are running any of the extra labs for creating a toolchain
they may overwrite this file, so beware that you are not running
the wrong cross-compiler. You could name it something else like
\code{yocto-toolchain.sh}

Now you can use the cross compiler by just sourcing this file.

\begin{verbatim}
source toolchain.sh
\end{verbatim}


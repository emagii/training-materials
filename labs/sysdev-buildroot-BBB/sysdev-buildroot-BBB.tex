\subchapter{Using a build system, example with Buildroot}{Objectives:
  discover how a build system is used and how it works, with the
  example of the Buildroot build system. Build a Linux system with
  libraries}

\section{Setup}

Go to the \code{$HOME/felabs/sysdev/buildroot/} directory,
which already contains some data needed for this lab, including a
kernel image.

\section{Get tarballs}
If you have a slow internet connection, and have a DVD with the needed tarballs
you can copy the \code{buildroot/download} directory on the DVD to 
\code{$HOME/felabs/sysdev/buildroot/download} 

\section{Get Buildroot and explore the source code}

The official Buildroot website is available at
\url{http://buildroot.org/}. Download the latest stable 2014.02.x
version which we have tested for this lab. Uncompress the tarball
and go inside the Buildroot source directory.

Several subdirectories or files are visible, the most important ones
are:

\begin{itemize}
\item \code{boot} contains the Makefiles and configuration items
  related to the compilation of common bootloaders (Grub, U-Boot,
  Barebox, etc.)
\item \code{configs} contains a set of predefined configurations,
  similar to the concept of defconfig in the kernel.
  There is luckily a version for the \devboard.
\item \code{docs} contains the documentation for Buildroot. You can
  start reading buildroot.html which is the main Buildroot
  documentation;
\item \code{fs} contains the code used to generate the various root
  filesystem image formats
\item \code{linux} contains the Makefile and configuration items
  related to the compilation of the Linux kernel
\item \code{Makefile} is the main Makefile that we will use to use
  Buildroot: everything works through Makefiles in Buildroot;
\item \code{package} is a directory that contains all the Makefiles,
  patches and configuration items to compile the userspace
  applications and libraries of your embedded Linux system. Have a
  look at various subdirectories and see what they contain;
\item \code{system} contains the root filesystem skeleton and the {\em
    device tables} used when a static \code{/dev} is used;
\item \code{toolchain} contains the Makefiles, patches and
  configuration items to generate the crosscompiling toolchain;
\end{itemize}

\section{Set the download directory}
If you got the tarballs from a DVD, link the buildroot
download directory \code{${TOPDIR}/dl} to your download directory.

\begin{verbatim}
ln -s ../download dl
\end{verbatim}


\section{Configure Buildroot}

In our case, we would like to:

\begin{itemize}
\item Having Buildroot generate a toolchain for us;
\item Generate an embedded Linux system for ARM;
\item Integrate Busybox in our embedded Linux system;
\item Integrate the target filesystem into both an ext4 filesystem
  image and a tarball
\end{itemize}

To run the configuration utility of Buildroot, simply run:

\begin{verbatim}
make beaglebone_defconfig
\end{verbatim}

Then we want to customize the build.

\begin{verbatim}
make menuconfig
\end{verbatim}

Set the following options:

\begin{itemize}
\item \code{Toolchain}
  \begin{itemize}
  \item enable \code{Toolchain has large file support?}, 
  \item enable \code{Toolchain has RPC support?}
  \item enable \code{WCHAR support?}
  \item enable \code{Toolchain has C++ support?}.
  \item enable \code{Build cross-gdb for the host}.
  \end{itemize}
\item \code{Target packages}
  \begin{itemize}
  \item Keep \code{BusyBox} (default version) and keep the Busybox
    configuration proposed by Buildroot;
  \end{itemize}
\item \code{Filesystem images}
  \begin{itemize}
  \item Select \code{ext2/3/4 root filesystem}
  \item Select \code{tar the root filesystem with bzip2}
  \end{itemize}
\end{itemize}

Exit the menuconfig interface. Your configuration has now been saved
to the \code{.config} file.


\section{Generate the embedded Linux system}

Just run:

\begin{verbatim}
make
\end{verbatim}

Buildroot will download, extract, configure, compile and install each
component of the embedded system.

All the compilation has taken place in the \code{output/} subdirectory. Let's
explore its contents:

\begin{itemize}

\item \code{build}, is the directory in which each component built by
  Buildroot is extract, and where the build actually takes place

\item \code{host}, is the directory where Buildroot installs some
  components for the host. As Buildroot doesn't want to depend on too
  many things installed in the developer machines, it installs some
  tools needed to compile the packages for the target. In our case it
  installed pkg-config (since the version of the host may be ancient)
  and tools to generate the root filesystem image (genext2fs,
  makedevs, fakeroot)

\item \code{images}, which contains the final images produced by
  Buildroot. In the normal case it's just an ext2 filesystem image and a
  tarball of the filesystem, but depending on the Buildroot
  configuration, there could also be a kernel image or a bootloader
  image. This is where we find \code{rootfs.tar} and
  \code{rootfs.ext2}, which are respectively the tarball and the ext2
  image of the generated root filesystem.

\item \code{staging}, which contains the “build” space of the target
  system. All the target libraries, with headers, documentation. It
  also contains the system headers and the C library, which in our
  case have been copied from the cross-compiling toolchain.

\item \code{target}, is the target root filesystem. All applications
  and libraries, usually stripped, are installed in this
  directory. However, it cannot be used directly as the root
  filesystem, as all the device files are missing: it is not possible
  to create them without being root, and Buildroot has a policy of not
  running anything as root.

\item \code{toolchain}, is the location where the toolchain is
  built.

\end{itemize}

\section{Run the generated system}

Create a directory where you extract the tarball.

\begin{verbatim}
sudo tar --numeric-owner -jxvf rootfs.tar.bz2
\end{verbatim}

Softlink \code{${HOME}/rootfs} to this place and make sure you restart the NFS server.

Copy the \code{zImage} and the device tree file to \code{/tftpboot}
and boot using NFS.

Use the \code{root} username, no password is required.






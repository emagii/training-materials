\subchapter{Setting up Ubuntu}{General Setup of Ubuntu}

\section{Install Synaptic}

\begin{verbatim}
$ sudo apt-get install synaptic
$ sudo su
$ synaptic&
\end{verbatim}

Start \code{synaptic} and install 


\begin{itemize}

\item \code{gnome}

\item \code{nautilus-open-terminal}

\item \code{git}

\item \code{git-core}

\item \code{samba}

\item \code{system-config-samba}

\end{itemize}

Log out

Click on the white \code{Ubuntu} button when you get the login screen
and select the \code{Gnome Classic} option. Log again.

\code{Gnome Classic} will from now be your default.

\section{Open a terminal}

Since you installed \code{nautilus-open-terminal}, you can open a terminal
by right clicking the mouse, and select the terminal.


\section{Make sure bash is the default shell}

The normal Ubuntu installation uses the \code{dash} shell which won't work.

Changed to the \code{bash} shell.

\begin{verbatim}
$ cd /bin/
$ ls -l sh
lrwxrwxrwx 1 root root 9 dec  6 15:25 sh -> /bin/dash
$ sudo unlink sh
$ sudo ln -s /bin/bash sh
$ ls -l sh
lrwxrwxrwx 1 root root 9 dec  6 15:25 sh -> /bin/bash
\end{verbatim}

\clearpage full-sysdev-labs.pdf 
\section{Generate ssh keys}

If you already have \code{rsa} keys in the \code{$HOME/.ssh} directory, 
you can skip this step.

If not, you generate the keys like this (Make sure you are not running as super-user)

\begin{verbatim}
$ ssh-keygen -t rsa
\end{verbatim}

Use the default location and provide a password (twice).

This will generate 

\begin{itemize}

\item Private Key: \code{$HOME/.ssh/id_rsa}

\item Public Key: \code{$HOME/.ssh/id_rsa.pub}

\end{itemize}

The {\bf Private Key} should {\bf never} be give out to anyone else.

\section{Update Ubuntu to the latest package versions}

\fbox{\begin{minipage}{\textwidth}
{\bfseries
Caution: Do not \code{upgrade} to \code{Ubuntu 14.04}
}
\end{minipage}}



Run the update manager, to update the machine.

This will take some time.


\begin{verbatim}
Program->System Tools->Administration->Update
\end{verbatim}

\section{Make sure pkg-config works}

Edit \code{${HOME}/.bashrc} and add

\begin{verbatim}
export PKG_CONFIG_PATH=/usr/lib/x86_64-linux-gnu/pkgconfig
\end{verbatim}



You probably have to restart the computer afterwards.

